%%%%%%%%%%%%%%%%%%%%%%%%%%%%%%%%%%%%%%%%%
% Medium Length Professional CV
% LaTeX Template
% Version 2.0 (8/5/13)
%
% This template has been downloaded from:
% http://www.LaTeXTemplates.com
%
% Original author:
% Trey Hunner (http://www.treyhunner.com/)
%
% Important note:
% This template requires the resume.cls file to be in the same directory as the
% .tex file. The resume.cls file provides the resume style used for structuring the
% document.
%
%%%%%%%%%%%%%%%%%%%%%%%%%%%%%%%%%%%%%%%%%

%----------------------------------------------------------------------------------------
%	PACKAGES AND OTHER DOCUMENT CONFIGURATIONS
%----------------------------------------------------------------------------------------

\documentclass{resume} % Use the custom resume.cls style

\usepackage[left=0.4 in,top=0.4in,right=0.4 in,bottom=0.4in]{geometry} % Document margins
\newcommand{\tab}[1]{\hspace{.2667\textwidth}\rlap{#1}} 
\newcommand{\itab}[1]{\hspace{0em}\rlap{#1}}
\name{S\Large{YLVIA} \huge{P. K}\Large{INOSIAN}} % Your name
\address{Utah State Biology Department \\ 5305 Old Main Hill, Logan, UT 84322} % Your address
\address{518.708.5827 \\ sylvia.kinosian@gmail.com}  % Your phone number and email

\begin{document}
%----------------------------------------------------------------------------------------
%	EDUCATION SECTION
%----------------------------------------------------------------------------------------

\begin{rSection}{Education}

\textbf{Ph.D.}, Ecology, candidate, Advisor: Paul Wolf \hfill August 2016 - present
\\
Utah State University (USU), Logan, UT 

\textbf{Bachelor of Science}, Forestry, \textit{summa cum laude} \hfill August 2011 - May 2015
\\ 
University of Vermont (UVM), Burlington, VT.
\end{rSection}

%%%%%%%%%%%%%%%%%%%%%%%%%%%%%%%%%%%

\begin{rSection}{RESEARCH INTERESTS \& career goals}  
% super sciencey one
% I am intrigued by the role of polyploidy in speciation and diversification across ferns and its role in dispersal and colonization of novel habitats. My undergraduate research focused on the influence of polyploidy on long-distance dispersal in the context of Hawaiian \textit{Polystichum} ferns. I am also interested in the patterns of land plant evolution at the genomic level, and hope to increase our scientific knowledge on this subject by researching pteridophytes, the evolutionary midpoint between the better-known bryophytes and seed plants. My dissertation research is an exploration of the systematics of the genus \textit{Ceratopteris} as well as population genetics and biogeography in the group. In the future, I would like to continue in academia with research and teaching, to help increase our knowledge of pteridophytes and also excite the next generation of scientists.

% more general one
I am interested broadly patterns of evolution and diversification across land plants. My undergraduate research focused on the island biogeography of Hawaiian \textit{Polystichum} ferns, trying to elucidate the number of colonization events from the Asian mainland. This study piqued my interest in researching ferns, the evolutionary midpoint between the better-known bryophytes and seed plants. It also inspired me to focus on under-studied groups, like ferns, and stress their importance to scientists and lay-people alike. My dissertation research is an exploration of the systematics of the fern genus \textit{Ceratopteris}, one species of which (\textit{C. richardii}) has been used as a model organism for several decades. I hope to contribute to the knowledge of this genus by delineating species boundaries, which will aid future studies and selection of potential model organisms. I am also interested in using herbarium resources for research, education, and public outreach. In the future, I would like to continue in academia with research and teaching, to help increase our knowledge of ferns and also excite the next generation of scientists.
\end{rSection}

%----------------------------------------------------------------------------------------
%	Professional Experience 
%----------------------------------------------------------------------------------------

\begin{rSection}{Professional Experience}
\textbf{Teaching Assistant}. USU Biology Department, \textit{Instructor}: Lauren Lucas. \hfill January - May 2018
\\
\textit{Course}: Biology II Laboratory
\\
Implemented weekly labs and held office hours for three lab sections.

\textbf{Co-Instructor/TA}. USU Biology Department, \textit{Instructor}: Paul Wolf. \hfill August - December 2017
\\
\textit{Course}: Plant Systematics and Diversity
\\
Designed and taught weekly labs; wrote and proctored exams; gave lectures throughout the semester. 

\textbf{Field Technician}. USU Quinney College of Natural Resources. \hfill May - August 2016
\\
\textit{Supervisor}: Karen Mock
\\
Assisted with a study on aspen regeneration, measuring the growth of aspen seedlings in plots on Cedar Mountain.
 
\textbf{Laboratory Technician}. UVM Plant Biology Department. \hfill July - December 2015
\\
\textit{Supervisor}: Michael Sundue
\\
Performed genomic DNA extractions for various projects as well as identification of herbarium specimens.

\textbf{Intern}. Vermont Urban and Community Forestry. Burlington, VT. \hfill January - May 2014 
\\
Raised community awareness of invasive insects and planned a UVM Arbor Day celebration.

\textbf{Intern}. The Land Stewardship Program. Burlington, VT. \hfill June - August 2013 
\\
Conducted property surveys to inform management decisions using field and GIS-based techniques.

\textbf{Peer Tutor}. UVM Learning Cooperative. \hfill Fall 2012 - Spring 2014
\\
Tutored students in a variety of subjects including Botany, Astronomy, and Anthropology.

\end{rSection}


%%%%%%%%%%%%%%%%%%%%%%%%%

\begin{rSection}{Professional Affiliations and Student Organizations}
\textbf{Member}. American Society of Plant Taxonomists. \hfill April 2018 - present
\\
\textbf{Member}. American Fern Society. \hfill January 2018 - present
\\
\textbf{President}. USU Cycling Team. \hfill August 2017 - present
\\
\textbf{Organizing Committee}. USU Biology Programming Club. \hfill January 2017 - present
\\
\textbf{Vice President}. UVM Forestry Club. \hfill January - May 2015
\end{rSection}

%----------------------------------------------------------------------------------------
%	Honors
%----------------------------------------------------------------------------------------

\begin{rSection}{HONORS \& AWARDS}
Asian Symposium of Ferns and Lycophytes 2018 Student Travel Grant (\$480) \hfill October 2018
\\
American Society of Plant Taxonomists Travel Grant - Botany 2018 (\$335) \hfill July 2018
\\
Joseph E. Greaves Endowed Scholarship, USU Biology Department (\$4550) \hfill April 2017
\\
Organization for Tropical Studies, Barbara Joe Hoshizaki Memorial Scholarship (\$500) \hfill January 2017
\\
National Science Foundation Graduate Research Fellowship Program (\$102,000) \hfill April 2016
\\
Award for Excellence in Plant Biology, UVM Plant Biology Department (\$250) \hfill May 2015
\\
W. R. Adams Award for Outstanding Academic Achievement in Forestry, UVM Rubenstein School of 
\\ \hspace*{1cm} Environment and Natural Resources (RSENR) \hfill May 2015
\\
Holcomb Natural Resource Prize, UVM RSENR \hfill May 2015
\\
Dale Bergdahl Scholarship Award, UVM RSENR (\$1000) \hfill May 2014
\\
Dean's Book Award for Outstanding Juniors, UVM RSENR \hfill May 2013
\\
Dean's List, UVM RSENR \hfill Fall 2011 - Spring 2015

\end{rSection} 

%%%%%%%%%%%%%%%%%%%%%%%%

\begin{rSection}{PRESENTATIONS \& Publications}
Jacob Suissa and \textbf{Sylvia P. Kinosian}. 2018. Botany 2018: Exploring Minnesota's Driftless Area, with Drifting Pteridologists. Fiddlehead Forum. 45(4): 73-77.

\textbf{Sylvia P. Kinosian}. Population structure analysis of the pan-tropical fern genus \textit{Ceratopteris} (Pteridaceae). Oral presentation, Asian Symposium of Ferns and Lycophytes. Taipei, Taiwan. October 18 2018.

John Thomson, Paul G. Wolf, \textbf{Sylvia P. Kinosian}, Zach Gompert, Joshua Der, Carol Rowe, Martin P. Schilling, Trish McLenachen, Peter Lockhart, and Lara Shepherd. 2018. Relationships among worldwide groups in the fern genus \textit{Pteridium} (bracken) based on nuclear genome markers. American Journal of Botany. \textit{In prep}.

\textbf{Sylvia P. Kinosian}. Zach Gompert, Paul G. Wolf, Joshua Der, Carol Rowe, Martin P. Schilling, Trish McLenachen, Peter Lockhart, Lara Shepherd, and John Thomson. Population admixture in the cosmopolitan fern genus \textit{Pteridium} (Dennstaeditaceae). Oral presentation, Botany. Rochester, MN. July 24 2018.

\textbf{S. P. Kinosian}, W. D. Pearse, and M. E. Barkworth. Spindle: SPecimen INformation Data capture and Label crEation. Contributed poster, Botany. Rochester, MN. July 23 2018.

Mary Barkworth, Paul G. Wolf, \textbf{Sylvia P. Kinosian}, Curtis Dyreson, Will Pearse, Ben Brandt, and Neil Cobb. 2017. The Value of Agricultural Voucher Specimens. Proceedings of TDWG 1: e19932.

\textbf{Sylvia P. Kinosian}. Zach Gompert, Paul G. Wolf, Joshua Der, Carol Rowe, Martin P. Schilling, Trish McLenachen, Peter Lockhart, Lara Shepherd, and Jobn Thomson. Population admixture in the cosmopolitan fern genus \textit{Pteridium} (Dennstaeditaceae). Contributed poster, Evolution Conference. Portland, OR. June 24 2017.

William D. Pearse, Maxwell J. Farrell, Konrad Hafen, Mallory Hagadorn, Spencer B. Hudson. \textbf{Sylvia P. Kinosian}, Ryan McCleary, Alexandre Rego, and Katie Welgarz. 2017 (\textit{in prep}). NATDB: An R package that downloads species trait data, but is Not A Trait Database. Available on GitHub: goo.gl/i9LkLQ

\textbf{Sylvia P. Kinosian}. Admixture Analysis of \textit{Pteridium} (Dennstaeditaceae) in a global context. Presentation, Tropical Ferns and Lycophytes Course. Organization for Tropical Studies. San Vito, Costa Rica. January 12 2017.

\textbf{S. P. Kinosian}, N. Patel, and D. S. Barrington. Insights into the Hawaiian \textit{Polystichum} (Dryopteridaceae) species. Contributed poster, Next Generation Pteridology. Smithsonian Institution, Washington D.C. June 1 2015.

\textbf{S. P. Kinosian}., N. Patel, and D. S. Barrington. Insights into the Hawaiian \textit{Polystichum} (Dryopteridaceae) species. Contributed poster, University of Vermont Student Research Conference. Burlington, VT. April 23 2015.
\end{rSection}

%----------------------------------------------------------------------------------------
%	Field & Teaching Experience 
%----------------------------------------------------------------------------------------

\begin{rSection}{Field COURSES \& Trainings}
\textbf{AIARE I}. American Institute for Avalanche Research and Education. Logan, UT. \hfill January 2018
\\
Learned how to make informed decisions and stay safe in avalanche terrain.

\textbf{Wilderness First Aid}. Desert Mountain Medicine. Logan, UT \hfill May 2017
\\
Trained in wilderness safety and emergency medical protocols.

\textbf{Tropical Ferns and Lycophytes}. Organization for Tropical Studies. Costa Rica. \hfill January 2017
\\
Course emphasizing field identification and systematics of tropical pteridophytes.

\textbf{Tropical Plant Systematics}. UVM Plant Biology Department. Costa Rica. \hfill January 2015
\\
Explored tropical land plant diversity at four field sites in Costa Rica.

\textbf{Study Abroad}. Round River Conservation Studies. Patagonia, Chile. \hfill September - December 2014
\\
Semester course on conservation biology and wildness survival resulting in student-lead research projects.

\textbf{Study Abroad}. Round River Conservation Studies. British Columbia, Canada. \hfill June - August 2014
\\
Summer course on conservation biology and traditional ecological knowledge in the Tlingit First Nation.

\end{rSection}

%%%%%%%%%%%%%%%%%%%%%%%%%%%%%%%%%%%%%%%%%%
\begin{rSection}{Skills}
\textbf{Languages}: English (primary), Spanish (conversational)
\\
\textbf{Programming Languages}: R (proficient), BASH (proficient), \LaTeX (proficient), Angular 6 (HTML \& Typescript) (familiar), Perl (familiar), Python (familiar), C / C++ (familiar)
\\
\textbf{Administrator} of the Wolf-Mock Lab High Performance Computer Workstation (Ubuntu OS)
\\
\textbf{Miscellaneous}: able to drive 4x4 manual vehicles in the field; familiar with using a DSLR digital camera with manual settings; proficient with dichotomous identification keys; chainsaw certified through Game of Logging course; AIARE I and WFA certified, experienced bicycle (road \& mountain) mechanic. 
\end{rSection}

\end{document}
